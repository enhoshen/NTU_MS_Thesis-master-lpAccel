\chapter{Proposed Architecture}
\label{ch:arch}
The architecture design is targeted to capture as much data reuse as possible, a three-level hierarchy on-chip buffer exploits almost every data reuseability in a regular convolution layer in the mean time saving power by logistically decreasing buffer size: data stay longer in the lower and smaller hierarchy buffer. As parallelism goes up, memory dispatching to each PE is not a task regarding bandwidth, but a huge burden to the control logic; with data rearrangement, we can efficiently transfer data between buffer and buffer, or buffer and PE with shifter instead of MUX, saving logic and potentially power. Finally we propose three micro-architecture dedicated to low-bit multiplication adder tree, able to operate on 1,2,4,8 bits signed and unsigned data, and an additional XNOR functionality\TODO{final choice}.
\textcolor{purple}{This chapter is your proof. You have shown in the previous chapters what the starting point is and why it is important to advance. This chapter describes how you advance and how you made choices. It also shows what (performance) results your choices lead to. Ideally, it identifies how certain choices influence the final performance.}
\section{System Overview}
\subsection{Dataflow}
\subsection{Buffer hierarchy}
\section{Architecture Overview}
\subsection{PE processing pipeline}
\subsection{Re-configurable arithmetic logic unit}
\subsection{Partial sum propagate path}
\subsection{Data dispatch shifter}
