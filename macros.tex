\renewcommand{\baselinestretch}{1}       % for squeezing the draft into the page limit, do not use

% Note
% Use \textsc{Name} to separate images, videos, dataset names from the main texts.

% =============================================================================
% Commonly used packages
% =============================================================================

% For removing Package ifpdf Error
%%%%%%\let\ifpdf\relax

% For removing LaTeX Font Warning
\usepackage{lmodern}

% Add Reference in Contents
\usepackage[nottoc]{tocbibind}

% Figures
%\usepackage{subcaption}
\usepackage{float}
%\usepackage[justification=raggedright]{caption}	% makes captions ragged right - thanks to Bryce Lobdell
\usepackage{lscape} % Useful for wide tables or figures.
\usepackage{makecell}
\usepackage{sidecap}

\graphicspath{{figures}{example}}

% Algorithm
\usepackage[lined,ruled,linesnumbered]{algorithm2e}
\usepackage{algorithmic}

% Table and list
\usepackage{booktabs} % Publication quality tables
\usepackage{multirow}
\usepackage{rotating} % sideways
\newcommand{\vergap}[1]{\renewcommand{\arraystretch}{#1}}
\newcommand{\horgap}[1]{\setlength{\tabcolsep}{#1}}
%\specialrule{width}{abovespace}{belowspace}
\newcommand{\dtoprule}{\specialrule{2pt}{0pt}{2pt}}
\newcommand{\dbottomrule}{\specialrule{2pt}{0pt}{\belowrulesep}}

\usepackage{paralist}
\usepackage{enumitem}

% Math
\usepackage{bm} % Make bold, italic math symbols
\usepackage{epsfig} % for figures
\usepackage{graphicx} % another package that works for figures
\usepackage{times}
%\usepackage{mathptmx}
\usepackage{mathtools}
\usepackage{textcomp, gensymb} % math symbol
\usepackage{amssymb,amsmath,amsfonts} % Short math guide for LaTeX ftp://ftp.ams.org/pub/tex/doc/amsmath/short-math-guide.pdf
\usepackage{siunitx} % SI units
\newcommand{\norm}[1]{\left\lVert#1\right\rVert}
\newcommand{\cp}[1]{\left[#1\right]_{\times}}

% Fonts
\usepackage{units}
\usepackage{color}

% Comments
\usepackage{comment}

% Hyperlinks
\usepackage{url} % Hyphenation of URLs.
\usepackage{xcolor}
\usepackage[backref=page]{hyperref}
\hypersetup{colorlinks,breaklinks,
            urlcolor=[rgb]{0.918,0,0.545},
            linkcolor=[rgb]{0.710,0.180,0.141},
            citecolor=[rgb]{0,0.545,0.447}}
\usepackage{bookmark}
%\usepackage[pagebackref=true,breaklinks=true,colorlinks,bookmarks=false]{hyperref} % remove letterpaper=true,
%
%\usepackage{slashbox}
\usepackage[table,caption=false]{xcolor}
%\usepackage{setspace}

% Better hyphenation
\usepackage{microtype}

% Appendix
\usepackage[toc,page]{appendix}

% =========================================
% Useful macros
% =========================================

% Latin abbreviations
\newcommand{\etal}{\textit{et al}.~} % ``and others'', ``and co-workers''
\newcommand{\eg}{e.g.,~} % ``for example''
\newcommand{\ie}{i.e.,~} % ``that is'', ``in other words''
\newcommand{\suchthat}{\, \mid \,}

% Math related
\DeclareMathOperator*{\argmin}{\arg\!\min}
\DeclareMathOperator*{\argmax}{\arg\!\max}
\DeclareMathOperator{\avg}{avg}
\DeclareMathOperator{\Tr}{Tr}

% Paragraph
\let\originalparagraph\paragraph
\renewcommand{\paragraph}[2][.]{\originalparagraph{#2#1}}

% Consistent margin adjustment for paragraphs, figures, and sections
\newlength\paramargin
\newlength\figmargin
\newlength\secmargin

\setlength{\secmargin}{0.0mm}
\setlength{\paramargin}{0.0mm}
\setlength{\figmargin}{0.0mm}

% References for figures, tables, equations, chapters, and sections
\newcommand{\chref}[1]{Chapter~\ref{ch:#1}}
\newcommand{\secref}[1]{Section~\ref{sec:#1}}
\newcommand{\figref}[1]{Figure~\ref{fig:#1}}
\newcommand{\tabref}[1]{Table~\ref{tab:#1}}
\newcommand{\eqnref}[1]{\eqref{eq:#1}}
\newcommand{\thmref}[1]{Theorem~\ref{#1}}
\newcommand{\prgref}[1]{Program~\ref{#1}}
\newcommand{\algref}[1]{Algorithm~\ref{#1}}
\newcommand{\clmref}[1]{Claim~\ref{#1}}
\newcommand{\lemref}[1]{Lemma~\ref{#1}}
\newcommand{\ptyref}[1]{Property~\ref{#1}}

% Comments
\long\def\ignorethis#1{}
\newcommand {\sychien}[1]{{\color{blue}\textbf{Po-Chen: }#1}\normalfont}
\newcommand {\coauthorA}[1]{{\color{red}\textbf{Co-author A: }#1}\normalfont}
\newcommand {\coauthorB}[1]{{\color{magenta}\textbf{Co-author B: }#1}\normalfont}
\newcommand {\TODO}[1]{{\textbf{\color{red}[TO-DO]\_#1}}}
\def\newtext#1{\textcolor{blue}{#1}}
\def\modtext#1{\textcolor{red}{#1}}

%\usepackage{ifthen}
%\ifthenelse{\equal{\final}{1}}
%{
%  \renewcommand{\sychien}[1]{}
%}
%{}

\newcommand{\tb}[1]{\textbf{#1}}
\newcommand{\mb}[1]{\mathbf{#1}}
\newcommand{\Paragraph}[1]{\noindent\textbf{#1}}

\newcommand{\jbox}[2]{
  \fbox{%
  	\begin{minipage}{#1}%
  		\hfill\vspace{#2}%
  	\end{minipage}%
  }
}

\newcommand{\jblock}[2]{%
  \begin{minipage}[t]{#1}\vspace{0cm}\centering%
  #2%
  \end{minipage}%
}
	
% Customized definition
\newcommand{\Ic}{\mathcal{I}_{c}}
\newcommand{\It}{\mathcal{I}_{t}}
\newcommand{\Ot}{\mathcal{O}_{t}}
\newcommand{\asin}{\mathrm{asin}}
\newcommand{\acos}{\mathrm{acos}}
\newcommand{\atan}{\mathrm{atan}}
\newcommand{\atanT}{\mathrm{atan2}}
\newcommand{\dotP}{\mathrm{dot}}
